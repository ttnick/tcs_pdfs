\documentclass[11pt, a4paper]{article}
\usepackage[left=4cm, right=4cm]{geometry}
\usepackage[utf8]{inputenc}
\usepackage[ngerman]{babel}
\usepackage{amsmath}
\usepackage{amsthm}
\usepackage{amsfonts}
\usepackage{amssymb}
\usepackage{mathtools}
\usepackage{graphicx}
\usepackage{tikz}
\usetikzlibrary{shapes.multipart}
\usepackage{listings}
\usepackage[justification=centering]{caption}
\usetikzlibrary{arrows, automata, graphs, shapes, petri, decorations.pathmorphing}
\parindent = 0pt
\let\emptyset\varnothing

% define environments
\theoremstyle{definition}
\newtheorem{definition}{Definition}
\newtheorem*{definition*}{Definition}

\theoremstyle{plain}
\newtheorem{theorem}[definition]{Theorem}
\newtheorem{lemma}[definition]{Lemma}

\renewcommand{\labelenumi}{(\roman{enumi})}

\author{Niklas Rieken}
\title{Von Pushdown-Automaten\\zu kontextfreien Grammatiken}


\begin{document}
\maketitle
Die folgende Definition aus der Vorlesung beschreibt, wie sich aus einem Pushdown-Automaten (PDA), der mit leerem Stack akzeptiert, eine kontextfreie Grammatik (CFG) konstruieren lässt, die die selbe Sprache erzeugt. 
\begin{definition*}
	Sei $\mathcal{A} = (Q, \Sigma, \Gamma, \Delta, q_0, Z_0)$ ein PDA, der mit leerem Stack akzeptiert. Die kontextfreie Grammatik $\mathcal{G}_\mathcal{A} = (N, \Sigma, P, S)$ sei wie folgt definiert:
	\begin{itemize}
		\item $S$ ist ein neues Startsymbol.
		\item Nichtterminale $N \coloneqq \{S\} \cup \{[pZq] \mid p, q\in Q, Z \in \Gamma\}$.
		\item Produktionsregeln
			\begin{align*}
				P &\coloneqq \{[pZq] \to \sigma [p_0Z_1p_1][p_1Z_2p_2]\ldots[p_{m-1}Z_mq] \mid\\
				&\quad\quad (p, \sigma, Z, Z_1 \ldots Z_m, p_0) \in \Delta, p_1, \ldots, p_{m-1}, q \in Q\}\\
				&\cup \{S \to [q_0Z_0q] \mid q \in Q\}.
			\end{align*}
	\end{itemize}
\end{definition*}
Das Verfahren ist eigentlich sehr einfach, allerdings ist die Funktionsweise nicht sehr intuitiv, da sich die Produktionsregeln der erzeugten Grammatik nicht so leicht interpretieren lassen (oder: sie lassen sich nicht so einfach von links nach rechts lesen).
Wir betrachten also einen PDA, der mit leerem Stack akzeptiert $\mathcal{A} = (Q, \Sigma, \Gamma, \Delta, q_0, Z_0)$ und konstruieren daraus die Grammatik $\mathcal{G_A} = (N, \Sigma, P, S)$ wie in der Definition beschrieben. Die entscheidene Kernidee des Algorithmus ist folgende: Das Nichtterminalsymbol $[pZq]$ repräsentiert einen Lauf von $p$ nach $q$ bei dem das Stacksymbol $Z$ aus dem Stack entfernt wird (d.h. wir kommen dem leeren Stack ''einen Schritt näher``).\\
Ein Ableitungsschritt in der Grammatik entspricht dabei nicht einfach einer Transition im PDA (außer im Fall von $[pZq] \to \sigma$, was der Transition $(p, \sigma, Z, \varepsilon, q)$ entspricht), sondern auch der \textit{Hoffnung} den Stack, der sich in der Transition $(p, \sigma, Z, Z_1 \ldots Z_m, p_0)$ aufbaut (für $m > 0$) wieder abzubauen. Die Nichtterminale $[pZq]$ \textit{hoffen} also, dass der PDA von $p$ mit Stackinhalt $Z\gamma$ nach $q$ kommt und dabei nur $\gamma$ auf dem Stack stehen lässt, wobei $\gamma \in \Gamma^\ast$. 
Wir gehen nun einmal auf die drei Arten der Produktionsregeln ein, hierfür sei $\sigma \in \Sigma \cup \{ \varepsilon \}$:
\begin{itemize}
	\item Mit dem Startsymbol $S$ haben wir folgende Produktionsregeln: $S \to [q_0 Z_0 q]$ \textbf{für alle} $q \in Q$.\\
	Die Akzeptanzbedingung des Automatenmodells ist allein vom Stack abhängig, in welchem Zustand ein Lauf terminiert ist also egal, deswegen bilden wir diese Regel für jedes $q$. Hier rufen wir uns nochmal die Intuition der Nichtterminale (außer $S$) in Erinnerung: Gesucht wird also ein Lauf auf $\mathcal{A}$ vom Startzustand $q_0$ zu einem beliebigen $q$, sodass das Bottom-of-Stack-Symbol $Z_0$ weggenommen wird (d.h. nur noch $\varepsilon$ auf dem Stack steht). Diese Regeln sind drücken also genau unseren gewünschten Lauf aus: Von der Startkonfiguration $(q_0, Z_0)$ zu einer Konfiguration $(q, \varepsilon)$.
	\item Für die Transitionen der Form $(p, \sigma, Z, \varepsilon, q)$ erhalten wir die Produktionsregel $[pZq] \to \sigma$. Dies sind die Regeln in denen der Stack tatsächlich abgebaut wird ohne ihn vorher zu erweitern (s. nächster Punkt).
	\item Regeln der Form $[pZp_m] \to \sigma [p_0 Z_1 p_1][p1 Z_2 p_2] \ldots [p_{m-1} Z_m p_m]$ für eine Transition $(p, \sigma, Z, Z_1 \ldots Z_m, p_0) \in \Delta$ und alle möglichen(!) Zustände $p_1, \ldots, p_m \in Q$:\\
	Dies ist die schwerste Art der konstruierten Produktionsregeln. Wir versuchen dies trotzdem einmal in einem Satz von links nach rechts zu beschreiben, auch wenn das etwas konstruiert wirkt:
	Wir befinden uns in Zustand $p$ mit $Z$ oben auf dem Stack, welches wir entfernen wollen, und dabei den Zustand $p_m$ erreichen mit einer Transition über $\sigma$, wo wir den Stack mit $Z_1, \ldots, Z_m$ auffüllen, welche wir dann in der Zukunft, d.h. in einem Lauf über $p_0$ bis $p_m$, abbauen müssen über weitere Regeln dieser Form und -- schlussendlich -- über Regeln der Art aus dem vorigen Punkt.\\
	Eine Ableitung eines Wortes repräsentiert also genau einen akzeptierenden (d.h. mit $\varepsilon$ auf dem Stack am Ende) Lauf auf dem PDA. In Abbildung~\ref{fig:stacks} ist die Idee nochmal an einer Transition veranschaulicht.
\end{itemize}
\begin{figure}
	\centering
	\begin{tikzpicture}[stack/.style={rectangle split, rectangle split parts=#1,draw, anchor=south}]
		\node[stack=6] at (0, 0) {
			\nodepart{one}$Z$
			\nodepart{two}$X_\ell$
			\nodepart{three}$\vdots$
			\nodepart{four}$X_2$
			\nodepart{five}$X_1$
			\nodepart{six}$Z_0$
		} node [below] {$p$};
		\node at (.75, 0) {$\to$};
		\node[stack=9] at (1.5, 0) {
			\nodepart{one}$Z_1$
			\nodepart{two}$Z_2$
			\nodepart{three}$\vdots$
			\nodepart{four}$Z_m$
			\nodepart{five}$X_\ell$
			\nodepart{six}$\vdots$
			\nodepart{seven}$X_2$
			\nodepart{eight}$X_1$
			\nodepart{nine}$Z_0$
		} node at (1.5, 0) [below] {$p_0$};
		\node at (2.25, 0) {$\to^\ast$};
		\node[stack=8] at (3, 0) {
			\nodepart{one}$Z_2$
			\nodepart{two}$\vdots$
			\nodepart{three}$Z_m$
			\nodepart{four}$X_\ell$
			\nodepart{five}$\vdots$
			\nodepart{six}$X_2$
			\nodepart{seven}$X_1$
			\nodepart{eight}$Z_0$
		} node at (3, 0) [below] {$p_1$};
		\node at (3.75, 0) {$\to^\ast$};
		\node at (4.5, 0) {$\cdots$};
		\node at (5.25, 0) {$\to^\ast$};
		\node[stack=6] at (6, 0) {
			\nodepart{one}$Z_m$
			\nodepart{two}$X_\ell$
			\nodepart{three}$\vdots$
			\nodepart{four}$X_2$
			\nodepart{five}$X_1$
			\nodepart{six}$Z_0$
		} node at (6, 0) [below] {$p_{m-1}$};
		\node at (6.75, 0) {$\to^\ast$};
		\node[stack=5] at (7.5, 0) {
			\nodepart{one}$X_\ell$
			\nodepart{two}$\vdots$
			\nodepart{three}$X_2$
			\nodepart{four}$X_1$
			\nodepart{five}$Z_0$
		} node at (7.5, 0) [below] {$p_m = q$};
	\end{tikzpicture}
	\caption{Ein Teillauf im PDA, welche durch das Nichtterminal $[pZq]$ simuliert wird. Zunächst benutzen wir die Transition $(p, \sigma, Z, Z_1 \ldots Z_m, p_0)$ um $Z$ zu entfernen, dadurch entstehen jedoch die Symbole $Z_1, \ldots, Z_m$ neu auf dem Stack, die im weiteren Verlauf abgebaut werden müssen. Dabei werden ist es für $[pZq]$ \textit{egal} mit welchen Zuständen $p_1, \ldots, p_{m-1}$ dies erreicht wird. Daher gibt es für jede $m$-Auswahl aus $Q$ eine Produktion. Beachte außerdem, dass die Schritte zwischen $p_i$ zu $p_{i+1}$ (für $0 \leq i < m$) \textit{Makro-Schritte} ($\to^\ast$ -- mehrere Transitionen) sind, d.h. nicht wie von $p$ nach $p_0$ eine einzelne Transition (nur durch Transition $(p_i, \sigma, Z_{i+1}, \varepsilon, p_{i+1})$), sondern durch weitere Zwischenschritte mit zwischenzeitlichem Aufbau des Stacks.}
	\label{fig:stacks}
\end{figure}
Um die Philosophie der Konstruktion einmal zusammenzufassen: Wenn in einem akzeptierenden Lauf der aktuelle Stackinhalt zu einem Zeitpunkt $Z\gamma$ ist, dann wird zu einem späteren Punkt des Laufes der Stackinhalt einmal nur noch $\gamma$ sein.\\
Wir betrachten als Beispiel den PDA, der mit leerem Stack akzeptiert in Abbildung~\ref{fig:pda}. Sei dafür 
$$
	\mathcal{A} = (\{ p, q \}, \{ a, b \}, \{ Z \}, \{(p, a, Z, ZZ, p), (p, a, Z, Z, q), (q, b, Z, \varepsilon, q) \}, p, Z). 
$$
\begin{figure}
	\centering
	\begin{tikzpicture}[->, >=stealth', shorten >=1pt, auto, semithick]
		\node[initial, state, initial text=] (p) at (0, 0) {$p$};
		\node[state] (q) at (4, 0) {$q$};
		
		\path (p) edge node {$(a, Z, Z)$} (q)
			(p) edge[loop above] node {$(a, Z, ZZ)$} (p)
			(q) edge[loop above] node {$(b, Z, \varepsilon)$} (q);		
	\end{tikzpicture}
	\caption{PDA, der mit leerem Stack akzeptiert für die Sprache $L = \{ a^n b^n : n \in \mathbb{N}_+ \}$.}
	\label{fig:pda}
\end{figure}
Wir erhalten zunächst für die Grammatik $\mathcal{G_A} = (N, \{ a, b \}, P, S)$ mit $N = \{ S, [pZp], [pZq], [qZp], [qZq] \}$. Die Produktionsregeln $P$ konstruieren wir nacheinander:
\begin{itemize}
	\item Zunächst ist es uns egal in welchem Zustand wir terminieren (auch wenn ein akzeptierender Lauf offensichtlich nur in $q$ enden kann), die einzige Bedingung, die wir an den Lauf stellen ist, dass er im Startzustand $p$ beginnt mit dem Bottom-of-Stack-Symbol $Z$ oben auf dem Stack. Insgesamt erhalten wir also die Regeln
	$$
		S \to [pZp] \,\vert\, [pZq] \in P.
	$$
	\item Nun betrachten wir die Transitionen, die den Stack abbauen. Davon gibt es nur eine im Loop über $q$, d.h. $(q, b, Z, \varepsilon, q)$. Wir erhalten also:
	$$
		[qZq] \to b \in P.
	$$
	\item Zuletzt noch die Transitionen, die den Stack vorübergehend aufbauen. Für die Transition $(p, a, Z, ZZ, p)$ erhalten wir die Regeln
	\begin{align*}
		[pZp] &\to a[pZp][pZp] \,\vert\, a[pZq][qZp],\\
		[pZq] &\to a[pZp][pZq] \,\vert\, a[pZq][qZq] \in P
	\end{align*}
	und für die Transition $(p, a, Z, Z, q)$ nur die Regel
	$$
		[pZq] \to a[qZq] \in P.
	$$
	\item Wir stellen fest, nachdem wir alle Regeln aus den Transitionen gebaut haben, dass die Regel  $[pZp] \to a[pZq][qZp]$ weggelassen werden kann, da wir keine Regel mit $[qZp]$ als linker Seite haben. Ein Anwenden dieser Regel könnte also nicht terminieren. Dies ist auch intuitiv klar, da ein Lauf von $q$ nach $p$ -- unabhängig vom Stackinhalt -- nicht möglich ist.
\end{itemize}
\end{document}
